% Options for packages loaded elsewhere
\PassOptionsToPackage{unicode}{hyperref}
\PassOptionsToPackage{hyphens}{url}
%
\documentclass[
]{article}
\usepackage{lmodern}
\usepackage{amssymb,amsmath}
\usepackage{ifxetex,ifluatex}
\ifnum 0\ifxetex 1\fi\ifluatex 1\fi=0 % if pdftex
  \usepackage[T1]{fontenc}
  \usepackage[utf8]{inputenc}
  \usepackage{textcomp} % provide euro and other symbols
\else % if luatex or xetex
  \usepackage{unicode-math}
  \defaultfontfeatures{Scale=MatchLowercase}
  \defaultfontfeatures[\rmfamily]{Ligatures=TeX,Scale=1}
\fi
% Use upquote if available, for straight quotes in verbatim environments
\IfFileExists{upquote.sty}{\usepackage{upquote}}{}
\IfFileExists{microtype.sty}{% use microtype if available
  \usepackage[]{microtype}
  \UseMicrotypeSet[protrusion]{basicmath} % disable protrusion for tt fonts
}{}
\makeatletter
\@ifundefined{KOMAClassName}{% if non-KOMA class
  \IfFileExists{parskip.sty}{%
    \usepackage{parskip}
  }{% else
    \setlength{\parindent}{0pt}
    \setlength{\parskip}{6pt plus 2pt minus 1pt}}
}{% if KOMA class
  \KOMAoptions{parskip=half}}
\makeatother
\usepackage{xcolor}
\IfFileExists{xurl.sty}{\usepackage{xurl}}{} % add URL line breaks if available
\IfFileExists{bookmark.sty}{\usepackage{bookmark}}{\usepackage{hyperref}}
\hypersetup{
  pdftitle={Why you need a multinomial roll for infection status},
  pdfauthor={Arseniy Khvorov},
  hidelinks,
  pdfcreator={LaTeX via pandoc}}
\urlstyle{same} % disable monospaced font for URLs
\usepackage[margin=1in]{geometry}
\usepackage{longtable,booktabs}
% Correct order of tables after \paragraph or \subparagraph
\usepackage{etoolbox}
\makeatletter
\patchcmd\longtable{\par}{\if@noskipsec\mbox{}\fi\par}{}{}
\makeatother
% Allow footnotes in longtable head/foot
\IfFileExists{footnotehyper.sty}{\usepackage{footnotehyper}}{\usepackage{footnote}}
\makesavenoteenv{longtable}
\usepackage{graphicx,grffile}
\makeatletter
\def\maxwidth{\ifdim\Gin@nat@width>\linewidth\linewidth\else\Gin@nat@width\fi}
\def\maxheight{\ifdim\Gin@nat@height>\textheight\textheight\else\Gin@nat@height\fi}
\makeatother
% Scale images if necessary, so that they will not overflow the page
% margins by default, and it is still possible to overwrite the defaults
% using explicit options in \includegraphics[width, height, ...]{}
\setkeys{Gin}{width=\maxwidth,height=\maxheight,keepaspectratio}
% Set default figure placement to htbp
\makeatletter
\def\fps@figure{htbp}
\makeatother
\setlength{\emergencystretch}{3em} % prevent overfull lines
\providecommand{\tightlist}{%
  \setlength{\itemsep}{0pt}\setlength{\parskip}{0pt}}
\setcounter{secnumdepth}{5}

\title{Why you need a multinomial roll for infection status}
\author{Arseniy Khvorov}
\date{11/12/2019}

\begin{document}
\maketitle

\hypertarget{how-its-supposed-to-work}{%
\section{How it's supposed to work}\label{how-its-supposed-to-work}}

\(F\) --- infected with flu\\
\(L\) --- infected with non-flu\\
\(V\) --- vaccinated\\
\(U\) --- unvaccinated\\
\(e\) --- vaccine effectiveness

\[
\begin{gathered}
P(F) = f \\
P(L) = l \\
P(V) = v \\
P(U) = 1 - v = u \\
P(F,V) = vf(1-e) \\
P(F,U) = uf \\
P(L,V) = vl \\
P(L,U) = ul \\
OR = \frac{P(F,V)P(L,U)}{P(F,U)P(L,V)}=\frac{vf(1-e)ul}{ufvl}=1-e
\end{gathered}
\]

\hypertarget{how-it-works-with-a-multinomial-roll}{%
\section{How it works with a multinomial roll}\label{how-it-works-with-a-multinomial-roll}}

\[
\begin{gathered}
P(F|V) = f(1-e) \\
P(F|U) = f \\
P(L|V) = l \\
P(L|U) = l \\
P(F,V) = P(F|V)P(V) = vf(1-e) \\
P(F,U) = P(F|U)P(U) = uf \\
P(L,V) = P(L|V)P(V) = vl \\
P(L,U) = P(L|U)P(U) = ul \\
OR = \frac{P(F,V)P(L,U)}{P(F,U)P(L,V)}=\frac{vf(1-e)ul}{ufvl}=1-e
\end{gathered}
\]

It works exaclty how it's supposed to.

\hypertarget{how-it-works-with-a-sequential-roll}{%
\section{How it works with a sequential roll}\label{how-it-works-with-a-sequential-roll}}

If the first roll works out flu infection and the second roll (non-flu infection) only applies to those not infected with flu, then

\[
\begin{gathered}
P(F|V) = f(1-e) \\
P(F|U) = f \\
P(L|V) = l(1 - f(1 - e)) \\
P(L|U) = l(1-f) \\
P(F,V) = P(F|V)P(V) = vf(1-e) \\
P(F,U) = P(F|U)P(U) = uf \\
P(L,V) = P(L|V)P(V) = vl(1 - f(1 - e)) \\
P(L,U) = P(L|U)P(U) = ul(1-f) \\
OR = \frac{P(F,V)P(L,U)}{P(F,U)P(L,V)}=\frac{vf(1-e)ul(1-f)}{ufvl(1 - f(1 - e))} = (1-e)\frac{1-f}{1-f(1-e)}
\end{gathered}
\]

The OR is biased.

Note that the ``central assumption'' is that \(P(L|V) = P(L|U)\) and with a sequential roll

\[
\begin{gathered}
P(L|V) = l(1 - f(1 - e)) \\
P(L|U) = l(1-f)
\end{gathered}
\]

the assumption is violated (unless \(e=0\)), in fact, vaccination increases the probability of non-flu (since \(f(1 - e) \leq f\)).

Also note that the total proportion infected with non-flu is

\[
\begin{aligned}
P(L,V) + P(L,U) &= vl(1 - f(1 - e)) + ul(1-f) \\  
&= l(v - vf(1-e) + (1-v)(1-f)) \\
&= l(v - vf(1-e) + 1 - f -v + vf) \\
&= l(vfe + 1 - f)
\end{aligned}
\]

is not \(l\) and the expected number of people infected with non-flu is not \(Nl\) where \(N\) is the total population size.

\end{document}
